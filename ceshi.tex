\documentclass{ctexart}
\usepackage[ustc,thu,whu]{cnlogo}
\setmainfont{Times New Roman}
\ctexset{punct=kaiming}
\usepackage{tabularx}
\newcolumntype{Y}{>{\centering\arraybackslash}X}
\usepackage{multirow,makecell}
\usepackage{geometry}
\geometry{
	a4paper,
	top=25.4mm, bottom=25.4mm,
	headheight=2.17cm,
	headsep=4mm,
	footskip=12mm
}
\usepackage{hyperref}
\hypersetup{pdftitle={cnlogo包使用手册},pdfauthor={向禹}}
\title{cnlogo包使用说明}\author{向禹}
\begin{document}
\maketitle
本宏包名叫cnlogo,是利用tikz绘制国内高校的校徽,借助学校本身提供的pdf高清矢量图通过inkscape转化为tikz代码。到目前为止已经绘制了60多所学校的校徽,有北京大学(pku),清华大学(thu),中国科学技术大学(ustc),复旦大学(fdu),上海交通大学(stju),武汉大学(whu),浙江大学(zju),厦门大学(xmu),西安交通大学(xjtu),南方科技大学(sustc),华中科技大学(hustc),同济大学(tju),北京交通大学(bjtu),湖南师范大学(hnnu),贵阳师范大学(gnu),中国政法大学(cupsl),华中师范大学(hznu),四川大学(scu),山东大学(sdu),吉林大学(jlu),中山大学(sysu),上海大学(shu),暨南大学(jnu),深圳大学(szu),湖南大学(hnu),中国矿业大学(ckyu),西安电子科技大学(xdu),温州大学(wzu),西南农业大学(swau),天津大学(tjuu),广东外语外贸大学(gufs),北京科技大学(ustb),中国人民大学(ruc),湖北大学(hbu),长安大学(chu),西南师范大学(swcnu),东北大学(neu),南开大学(nku),南京大学,东北财经大学(dbcju),大连理工大学(dllgu),江南大学(ju),中国地质大学(cug),中央民族大学(muc),河海大学(hhu),华东政法大学(hdzfu),中国农业大学(znyu),山东财经大学(sdcju),扬州大学(yzu),北京师范大学(bnu),郑州大学(zzu),北京邮电大学(bjydu),南昌大学(ncu),北京航空航天大学(bhu),华东师范大学(hdsfu),中国海洋大学(zhyu),中南大学(csu),兰州大学(lzu),华南理工大学(hnlgu),西北农林科技大学(xbnlkju),安徽大学与合工大(ahu),武汉理工大学(whlgu),中国传媒大学(zcmu),哈尔滨工业大学(hiu),中国科学院(zky)。


大家下载这个包可以直接使用,或者将它安装到你的\TeX 发行版中。以\TeX live为例,将整个文件夹复制到\texttt{texlive/texmf-local/tex/latex/}下,然后命令行执行texhash就可以完成安装,此后要想使用这个包,就可以直接在导言区输入\verb|\usepackage[选项]{cnlogo}|即可,这里的选项有pku,thu,ustc,fdu,stju,whu,zju,xmu,xjtu,sustcd等,分别对应各个学校。可以加载一个或多个选项,比如\verb|\usepackage[ustc,thu,pku]{cnlogo}|就可以输入与科大,清华,北大相关的命令。如果不加可选项则为空包。

本包的命令都以学校名为前缀,后缀为logo,text,whole或者side,这里的logo表示校徽,text表示校名,whole和side表示校徽和校名都有。每个命令包含$2\sim5$个可选参数,其中最后一个可选参数是scale的大小,只需要在可选框中输入一个数字即可,表示整个图像的放缩比例,默认值均为1,本包自定义了很多颜色名,都是用对应的学校名称命名的,因为在获取矢量图的时候的颜色就是如此。

我们将部分命令及其对应的可选参数和默认值以表格形式呈现:

{
\centering
\begin{tabularx}{\textwidth}{|c|c|Y|}
\hline
学校&命令&说明\\
\hline
\multirow{2}*{复旦}&\verb|\fdulogo[black][1]|&\multirowcell{2}{两个可选参数,首参数默认颜色black,\\第二个参数是scale,默认值为1}\\\cline{2-2}
                   &\verb|\fdutext[black][1]|&\\
\hline
\multirow{3}*{上交}&\verb|\stjulogo[stju1][1]|&\multirowcell{3}{两个可选参数,首参数默认\\颜色stju1和stju2已定义}\\\cline{2-2}
                  &\verb|\stjutext[stju1][1]|&\\\cline{2-2}
                  &\verb|\stjuside[stju1][1]|&\\
\hline
\multirow{2}*{南科}&\verb|\sustclogocn[sustc1][sustc2][1]|&\multirowcell{2}{三个可选参数,cn和en分别对应\\ 中文和英文logo}\\\cline{2-2}
                   &\verb|\sustclogoen[sustc1][sustc2][1]|&\\
\hline
\multirow{4}*{清华}&\verb|\thulogo[thu][1]|&\multirowcell{4}{两个可选参数,其中\verb|\thulib|\\是清华大学图书馆logo}\\\cline{2-2}
                  &\verb|\thulib[thu][1]|&\\\cline{2-2}
                  &\verb|\thutext[thu][1]|&\\\cline{2-2}
                  &\verb|\thuside[thu][1]|&\\
\hline
\multirow{4}*{科大}&\verb|\ustclogo[ustc][1]|&\multirowcell{4}{两个可选参数,其中\verb|\ustcside|,\\\verb|\ustcwhole|都是text+logo,形状不同}\\\cline{2-2}
                  &\verb|\ustcwhole[ustc][1]|&\\\cline{2-2}
                  &\verb|\ustctext[ustc][1]|&\\\cline{2-2}
                  &\verb|\ustcside[ustc][1]|&\\
\hline
学校&命令&说明\\
\hline
北大&\verb|\pkutext[pku][1]|&两个可选参数,首参数默认颜色pku已定义,第二个参数是scale,默认值为1\\
\hline
西交&\verb|\xjtutext[pku][1]|&\\
\hline
武大&\verb|\whulogo[whu1][whu2][whu3][whu4][1]|&五个参数,前四个是颜色\\
\hline
\multirow{2}*{浙大}&\verb|\zjulogo[zju][1]|&\multirowcell{2}{}\\\cline{2-2}
                  &\verb|\zjutext[zju][1]|&\\
\hline
厦大&\verb|\xmutext[xmu][1]|&\\
\hline
\end{tabularx}
}

以科大的四个命令为例,\verb|\ustclogo|带有两个可选参数,第一个参数是图的颜色,默认值为\verb|ustc|,其中颜色\verb|ustc|的定义为\verb|\definecolor{ustc}{RGB}{37,74,165}|。输入\verb|\ustclogo|和 \verb|\ustclogo[ustc][1]|是等价的,都是
\begin{center}
\ustclogo
\end{center}

我们可以更改可选参数,比如分别输入\verb|\ustclogo[green!40!black][0.2]|和\verb|\ustclogo[red]|就得到
\begin{center}
\ustclogo[green!40!black][0.2]
\end{center}
\begin{center}
\ustclogo[red]
\end{center}

剩下的命令都是同理,输入\verb|\ustctext[green!40!black][0.2]|,\verb|\ustcside[green!40!black][0.2]|,\verb|\ustcwhole[green!40!black][0.2]|分别得到
\begin{center}
\ustctext[green!40!black][0.2]\ustcside[green!40!black][0.2]\ustcwhole[green!40!black][0.2]
\end{center}

其中稍微麻烦一点的是武大的logo,带有四个可选的颜色,南科大的中英文logo带三个可选参数,颜色参数有两个,使用方法同理,我们建议大家先按照自己喜欢的颜色生成pdf图片,然后再将pdf图片插入文档。比如
\begin{verbatim}
\documentclass{article}
\usepackage{tikz}
\usepackage[active,tightpage]{preview}
\PreviewEnvironment{tikzpicture}
\usepackage[whu]{cnlogo}
\begin{document}
\whulogo
\end{document}
\end{verbatim}
先生成武大logo的pdf,然后再将此图片进行插图。

后面的学校及其相关命令太多,这里就不再列举,大家如果需要对应学校的logo,比如说哈尔滨工业大学(hiu),就查看hiu.tex中所定义的命令即可,我们尽可能提供是三个完整的命令,但是对于很多学校可能值提供了一个logo命令,这里不再详细介绍了。
\end{document} 